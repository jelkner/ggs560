
% Default to the notebook output style

    


% Inherit from the specified cell style.




    
\documentclass[11pt]{article}

    
    
    \usepackage[T1]{fontenc}
    % Nicer default font (+ math font) than Computer Modern for most use cases
    \usepackage{mathpazo}

    % Basic figure setup, for now with no caption control since it's done
    % automatically by Pandoc (which extracts ![](path) syntax from Markdown).
    \usepackage{graphicx}
    % We will generate all images so they have a width \maxwidth. This means
    % that they will get their normal width if they fit onto the page, but
    % are scaled down if they would overflow the margins.
    \makeatletter
    \def\maxwidth{\ifdim\Gin@nat@width>\linewidth\linewidth
    \else\Gin@nat@width\fi}
    \makeatother
    \let\Oldincludegraphics\includegraphics
    % Set max figure width to be 80% of text width, for now hardcoded.
    \renewcommand{\includegraphics}[1]{\Oldincludegraphics[width=.8\maxwidth]{#1}}
    % Ensure that by default, figures have no caption (until we provide a
    % proper Figure object with a Caption API and a way to capture that
    % in the conversion process - todo).
    \usepackage{caption}
    \DeclareCaptionLabelFormat{nolabel}{}
    \captionsetup{labelformat=nolabel}

    \usepackage{adjustbox} % Used to constrain images to a maximum size 
    \usepackage{xcolor} % Allow colors to be defined
    \usepackage{enumerate} % Needed for markdown enumerations to work
    \usepackage{geometry} % Used to adjust the document margins
    \usepackage{amsmath} % Equations
    \usepackage{amssymb} % Equations
    \usepackage{textcomp} % defines textquotesingle
    % Hack from http://tex.stackexchange.com/a/47451/13684:
    \AtBeginDocument{%
        \def\PYZsq{\textquotesingle}% Upright quotes in Pygmentized code
    }
    \usepackage{upquote} % Upright quotes for verbatim code
    \usepackage{eurosym} % defines \euro
    \usepackage[mathletters]{ucs} % Extended unicode (utf-8) support
    \usepackage[utf8x]{inputenc} % Allow utf-8 characters in the tex document
    \usepackage{fancyvrb} % verbatim replacement that allows latex
    \usepackage{grffile} % extends the file name processing of package graphics 
                         % to support a larger range 
    % The hyperref package gives us a pdf with properly built
    % internal navigation ('pdf bookmarks' for the table of contents,
    % internal cross-reference links, web links for URLs, etc.)
    \usepackage{hyperref}
    \usepackage{longtable} % longtable support required by pandoc >1.10
    \usepackage{booktabs}  % table support for pandoc > 1.12.2
    \usepackage[inline]{enumitem} % IRkernel/repr support (it uses the enumerate* environment)
    \usepackage[normalem]{ulem} % ulem is needed to support strikethroughs (\sout)
                                % normalem makes italics be italics, not underlines
    

    
    
    % Colors for the hyperref package
    \definecolor{urlcolor}{rgb}{0,.145,.698}
    \definecolor{linkcolor}{rgb}{.71,0.21,0.01}
    \definecolor{citecolor}{rgb}{.12,.54,.11}

    % ANSI colors
    \definecolor{ansi-black}{HTML}{3E424D}
    \definecolor{ansi-black-intense}{HTML}{282C36}
    \definecolor{ansi-red}{HTML}{E75C58}
    \definecolor{ansi-red-intense}{HTML}{B22B31}
    \definecolor{ansi-green}{HTML}{00A250}
    \definecolor{ansi-green-intense}{HTML}{007427}
    \definecolor{ansi-yellow}{HTML}{DDB62B}
    \definecolor{ansi-yellow-intense}{HTML}{B27D12}
    \definecolor{ansi-blue}{HTML}{208FFB}
    \definecolor{ansi-blue-intense}{HTML}{0065CA}
    \definecolor{ansi-magenta}{HTML}{D160C4}
    \definecolor{ansi-magenta-intense}{HTML}{A03196}
    \definecolor{ansi-cyan}{HTML}{60C6C8}
    \definecolor{ansi-cyan-intense}{HTML}{258F8F}
    \definecolor{ansi-white}{HTML}{C5C1B4}
    \definecolor{ansi-white-intense}{HTML}{A1A6B2}

    % commands and environments needed by pandoc snippets
    % extracted from the output of `pandoc -s`
    \providecommand{\tightlist}{%
      \setlength{\itemsep}{0pt}\setlength{\parskip}{0pt}}
    \DefineVerbatimEnvironment{Highlighting}{Verbatim}{commandchars=\\\{\}}
    % Add ',fontsize=\small' for more characters per line
    \newenvironment{Shaded}{}{}
    \newcommand{\KeywordTok}[1]{\textcolor[rgb]{0.00,0.44,0.13}{\textbf{{#1}}}}
    \newcommand{\DataTypeTok}[1]{\textcolor[rgb]{0.56,0.13,0.00}{{#1}}}
    \newcommand{\DecValTok}[1]{\textcolor[rgb]{0.25,0.63,0.44}{{#1}}}
    \newcommand{\BaseNTok}[1]{\textcolor[rgb]{0.25,0.63,0.44}{{#1}}}
    \newcommand{\FloatTok}[1]{\textcolor[rgb]{0.25,0.63,0.44}{{#1}}}
    \newcommand{\CharTok}[1]{\textcolor[rgb]{0.25,0.44,0.63}{{#1}}}
    \newcommand{\StringTok}[1]{\textcolor[rgb]{0.25,0.44,0.63}{{#1}}}
    \newcommand{\CommentTok}[1]{\textcolor[rgb]{0.38,0.63,0.69}{\textit{{#1}}}}
    \newcommand{\OtherTok}[1]{\textcolor[rgb]{0.00,0.44,0.13}{{#1}}}
    \newcommand{\AlertTok}[1]{\textcolor[rgb]{1.00,0.00,0.00}{\textbf{{#1}}}}
    \newcommand{\FunctionTok}[1]{\textcolor[rgb]{0.02,0.16,0.49}{{#1}}}
    \newcommand{\RegionMarkerTok}[1]{{#1}}
    \newcommand{\ErrorTok}[1]{\textcolor[rgb]{1.00,0.00,0.00}{\textbf{{#1}}}}
    \newcommand{\NormalTok}[1]{{#1}}
    
    % Additional commands for more recent versions of Pandoc
    \newcommand{\ConstantTok}[1]{\textcolor[rgb]{0.53,0.00,0.00}{{#1}}}
    \newcommand{\SpecialCharTok}[1]{\textcolor[rgb]{0.25,0.44,0.63}{{#1}}}
    \newcommand{\VerbatimStringTok}[1]{\textcolor[rgb]{0.25,0.44,0.63}{{#1}}}
    \newcommand{\SpecialStringTok}[1]{\textcolor[rgb]{0.73,0.40,0.53}{{#1}}}
    \newcommand{\ImportTok}[1]{{#1}}
    \newcommand{\DocumentationTok}[1]{\textcolor[rgb]{0.73,0.13,0.13}{\textit{{#1}}}}
    \newcommand{\AnnotationTok}[1]{\textcolor[rgb]{0.38,0.63,0.69}{\textbf{\textit{{#1}}}}}
    \newcommand{\CommentVarTok}[1]{\textcolor[rgb]{0.38,0.63,0.69}{\textbf{\textit{{#1}}}}}
    \newcommand{\VariableTok}[1]{\textcolor[rgb]{0.10,0.09,0.49}{{#1}}}
    \newcommand{\ControlFlowTok}[1]{\textcolor[rgb]{0.00,0.44,0.13}{\textbf{{#1}}}}
    \newcommand{\OperatorTok}[1]{\textcolor[rgb]{0.40,0.40,0.40}{{#1}}}
    \newcommand{\BuiltInTok}[1]{{#1}}
    \newcommand{\ExtensionTok}[1]{{#1}}
    \newcommand{\PreprocessorTok}[1]{\textcolor[rgb]{0.74,0.48,0.00}{{#1}}}
    \newcommand{\AttributeTok}[1]{\textcolor[rgb]{0.49,0.56,0.16}{{#1}}}
    \newcommand{\InformationTok}[1]{\textcolor[rgb]{0.38,0.63,0.69}{\textbf{\textit{{#1}}}}}
    \newcommand{\WarningTok}[1]{\textcolor[rgb]{0.38,0.63,0.69}{\textbf{\textit{{#1}}}}}
    
    
    % Define a nice break command that doesn't care if a line doesn't already
    % exist.
    \def\br{\hspace*{\fill} \\* }
    % Math Jax compatability definitions
    \def\gt{>}
    \def\lt{<}
    % Document parameters
    \title{GGS560 Homework 9}
    
    
    

    % Pygments definitions
    
\makeatletter
\def\PY@reset{\let\PY@it=\relax \let\PY@bf=\relax%
    \let\PY@ul=\relax \let\PY@tc=\relax%
    \let\PY@bc=\relax \let\PY@ff=\relax}
\def\PY@tok#1{\csname PY@tok@#1\endcsname}
\def\PY@toks#1+{\ifx\relax#1\empty\else%
    \PY@tok{#1}\expandafter\PY@toks\fi}
\def\PY@do#1{\PY@bc{\PY@tc{\PY@ul{%
    \PY@it{\PY@bf{\PY@ff{#1}}}}}}}
\def\PY#1#2{\PY@reset\PY@toks#1+\relax+\PY@do{#2}}

\expandafter\def\csname PY@tok@sh\endcsname{\def\PY@tc##1{\textcolor[rgb]{0.73,0.13,0.13}{##1}}}
\expandafter\def\csname PY@tok@mb\endcsname{\def\PY@tc##1{\textcolor[rgb]{0.40,0.40,0.40}{##1}}}
\expandafter\def\csname PY@tok@sd\endcsname{\let\PY@it=\textit\def\PY@tc##1{\textcolor[rgb]{0.73,0.13,0.13}{##1}}}
\expandafter\def\csname PY@tok@il\endcsname{\def\PY@tc##1{\textcolor[rgb]{0.40,0.40,0.40}{##1}}}
\expandafter\def\csname PY@tok@nt\endcsname{\let\PY@bf=\textbf\def\PY@tc##1{\textcolor[rgb]{0.00,0.50,0.00}{##1}}}
\expandafter\def\csname PY@tok@kn\endcsname{\let\PY@bf=\textbf\def\PY@tc##1{\textcolor[rgb]{0.00,0.50,0.00}{##1}}}
\expandafter\def\csname PY@tok@se\endcsname{\let\PY@bf=\textbf\def\PY@tc##1{\textcolor[rgb]{0.73,0.40,0.13}{##1}}}
\expandafter\def\csname PY@tok@vc\endcsname{\def\PY@tc##1{\textcolor[rgb]{0.10,0.09,0.49}{##1}}}
\expandafter\def\csname PY@tok@gp\endcsname{\let\PY@bf=\textbf\def\PY@tc##1{\textcolor[rgb]{0.00,0.00,0.50}{##1}}}
\expandafter\def\csname PY@tok@err\endcsname{\def\PY@bc##1{\setlength{\fboxsep}{0pt}\fcolorbox[rgb]{1.00,0.00,0.00}{1,1,1}{\strut ##1}}}
\expandafter\def\csname PY@tok@mf\endcsname{\def\PY@tc##1{\textcolor[rgb]{0.40,0.40,0.40}{##1}}}
\expandafter\def\csname PY@tok@ge\endcsname{\let\PY@it=\textit}
\expandafter\def\csname PY@tok@cs\endcsname{\let\PY@it=\textit\def\PY@tc##1{\textcolor[rgb]{0.25,0.50,0.50}{##1}}}
\expandafter\def\csname PY@tok@na\endcsname{\def\PY@tc##1{\textcolor[rgb]{0.49,0.56,0.16}{##1}}}
\expandafter\def\csname PY@tok@vm\endcsname{\def\PY@tc##1{\textcolor[rgb]{0.10,0.09,0.49}{##1}}}
\expandafter\def\csname PY@tok@sa\endcsname{\def\PY@tc##1{\textcolor[rgb]{0.73,0.13,0.13}{##1}}}
\expandafter\def\csname PY@tok@kc\endcsname{\let\PY@bf=\textbf\def\PY@tc##1{\textcolor[rgb]{0.00,0.50,0.00}{##1}}}
\expandafter\def\csname PY@tok@gh\endcsname{\let\PY@bf=\textbf\def\PY@tc##1{\textcolor[rgb]{0.00,0.00,0.50}{##1}}}
\expandafter\def\csname PY@tok@nf\endcsname{\def\PY@tc##1{\textcolor[rgb]{0.00,0.00,1.00}{##1}}}
\expandafter\def\csname PY@tok@w\endcsname{\def\PY@tc##1{\textcolor[rgb]{0.73,0.73,0.73}{##1}}}
\expandafter\def\csname PY@tok@c\endcsname{\let\PY@it=\textit\def\PY@tc##1{\textcolor[rgb]{0.25,0.50,0.50}{##1}}}
\expandafter\def\csname PY@tok@si\endcsname{\let\PY@bf=\textbf\def\PY@tc##1{\textcolor[rgb]{0.73,0.40,0.53}{##1}}}
\expandafter\def\csname PY@tok@ne\endcsname{\let\PY@bf=\textbf\def\PY@tc##1{\textcolor[rgb]{0.82,0.25,0.23}{##1}}}
\expandafter\def\csname PY@tok@sx\endcsname{\def\PY@tc##1{\textcolor[rgb]{0.00,0.50,0.00}{##1}}}
\expandafter\def\csname PY@tok@ss\endcsname{\def\PY@tc##1{\textcolor[rgb]{0.10,0.09,0.49}{##1}}}
\expandafter\def\csname PY@tok@nb\endcsname{\def\PY@tc##1{\textcolor[rgb]{0.00,0.50,0.00}{##1}}}
\expandafter\def\csname PY@tok@cp\endcsname{\def\PY@tc##1{\textcolor[rgb]{0.74,0.48,0.00}{##1}}}
\expandafter\def\csname PY@tok@m\endcsname{\def\PY@tc##1{\textcolor[rgb]{0.40,0.40,0.40}{##1}}}
\expandafter\def\csname PY@tok@s2\endcsname{\def\PY@tc##1{\textcolor[rgb]{0.73,0.13,0.13}{##1}}}
\expandafter\def\csname PY@tok@o\endcsname{\def\PY@tc##1{\textcolor[rgb]{0.40,0.40,0.40}{##1}}}
\expandafter\def\csname PY@tok@s\endcsname{\def\PY@tc##1{\textcolor[rgb]{0.73,0.13,0.13}{##1}}}
\expandafter\def\csname PY@tok@cpf\endcsname{\let\PY@it=\textit\def\PY@tc##1{\textcolor[rgb]{0.25,0.50,0.50}{##1}}}
\expandafter\def\csname PY@tok@bp\endcsname{\def\PY@tc##1{\textcolor[rgb]{0.00,0.50,0.00}{##1}}}
\expandafter\def\csname PY@tok@vg\endcsname{\def\PY@tc##1{\textcolor[rgb]{0.10,0.09,0.49}{##1}}}
\expandafter\def\csname PY@tok@nc\endcsname{\let\PY@bf=\textbf\def\PY@tc##1{\textcolor[rgb]{0.00,0.00,1.00}{##1}}}
\expandafter\def\csname PY@tok@kp\endcsname{\def\PY@tc##1{\textcolor[rgb]{0.00,0.50,0.00}{##1}}}
\expandafter\def\csname PY@tok@sc\endcsname{\def\PY@tc##1{\textcolor[rgb]{0.73,0.13,0.13}{##1}}}
\expandafter\def\csname PY@tok@nn\endcsname{\let\PY@bf=\textbf\def\PY@tc##1{\textcolor[rgb]{0.00,0.00,1.00}{##1}}}
\expandafter\def\csname PY@tok@gi\endcsname{\def\PY@tc##1{\textcolor[rgb]{0.00,0.63,0.00}{##1}}}
\expandafter\def\csname PY@tok@cm\endcsname{\let\PY@it=\textit\def\PY@tc##1{\textcolor[rgb]{0.25,0.50,0.50}{##1}}}
\expandafter\def\csname PY@tok@sr\endcsname{\def\PY@tc##1{\textcolor[rgb]{0.73,0.40,0.53}{##1}}}
\expandafter\def\csname PY@tok@no\endcsname{\def\PY@tc##1{\textcolor[rgb]{0.53,0.00,0.00}{##1}}}
\expandafter\def\csname PY@tok@ch\endcsname{\let\PY@it=\textit\def\PY@tc##1{\textcolor[rgb]{0.25,0.50,0.50}{##1}}}
\expandafter\def\csname PY@tok@nv\endcsname{\def\PY@tc##1{\textcolor[rgb]{0.10,0.09,0.49}{##1}}}
\expandafter\def\csname PY@tok@mi\endcsname{\def\PY@tc##1{\textcolor[rgb]{0.40,0.40,0.40}{##1}}}
\expandafter\def\csname PY@tok@mh\endcsname{\def\PY@tc##1{\textcolor[rgb]{0.40,0.40,0.40}{##1}}}
\expandafter\def\csname PY@tok@fm\endcsname{\def\PY@tc##1{\textcolor[rgb]{0.00,0.00,1.00}{##1}}}
\expandafter\def\csname PY@tok@kr\endcsname{\let\PY@bf=\textbf\def\PY@tc##1{\textcolor[rgb]{0.00,0.50,0.00}{##1}}}
\expandafter\def\csname PY@tok@gu\endcsname{\let\PY@bf=\textbf\def\PY@tc##1{\textcolor[rgb]{0.50,0.00,0.50}{##1}}}
\expandafter\def\csname PY@tok@c1\endcsname{\let\PY@it=\textit\def\PY@tc##1{\textcolor[rgb]{0.25,0.50,0.50}{##1}}}
\expandafter\def\csname PY@tok@gs\endcsname{\let\PY@bf=\textbf}
\expandafter\def\csname PY@tok@nl\endcsname{\def\PY@tc##1{\textcolor[rgb]{0.63,0.63,0.00}{##1}}}
\expandafter\def\csname PY@tok@kd\endcsname{\let\PY@bf=\textbf\def\PY@tc##1{\textcolor[rgb]{0.00,0.50,0.00}{##1}}}
\expandafter\def\csname PY@tok@ni\endcsname{\let\PY@bf=\textbf\def\PY@tc##1{\textcolor[rgb]{0.60,0.60,0.60}{##1}}}
\expandafter\def\csname PY@tok@gd\endcsname{\def\PY@tc##1{\textcolor[rgb]{0.63,0.00,0.00}{##1}}}
\expandafter\def\csname PY@tok@sb\endcsname{\def\PY@tc##1{\textcolor[rgb]{0.73,0.13,0.13}{##1}}}
\expandafter\def\csname PY@tok@s1\endcsname{\def\PY@tc##1{\textcolor[rgb]{0.73,0.13,0.13}{##1}}}
\expandafter\def\csname PY@tok@go\endcsname{\def\PY@tc##1{\textcolor[rgb]{0.53,0.53,0.53}{##1}}}
\expandafter\def\csname PY@tok@kt\endcsname{\def\PY@tc##1{\textcolor[rgb]{0.69,0.00,0.25}{##1}}}
\expandafter\def\csname PY@tok@k\endcsname{\let\PY@bf=\textbf\def\PY@tc##1{\textcolor[rgb]{0.00,0.50,0.00}{##1}}}
\expandafter\def\csname PY@tok@gr\endcsname{\def\PY@tc##1{\textcolor[rgb]{1.00,0.00,0.00}{##1}}}
\expandafter\def\csname PY@tok@gt\endcsname{\def\PY@tc##1{\textcolor[rgb]{0.00,0.27,0.87}{##1}}}
\expandafter\def\csname PY@tok@vi\endcsname{\def\PY@tc##1{\textcolor[rgb]{0.10,0.09,0.49}{##1}}}
\expandafter\def\csname PY@tok@nd\endcsname{\def\PY@tc##1{\textcolor[rgb]{0.67,0.13,1.00}{##1}}}
\expandafter\def\csname PY@tok@dl\endcsname{\def\PY@tc##1{\textcolor[rgb]{0.73,0.13,0.13}{##1}}}
\expandafter\def\csname PY@tok@mo\endcsname{\def\PY@tc##1{\textcolor[rgb]{0.40,0.40,0.40}{##1}}}
\expandafter\def\csname PY@tok@ow\endcsname{\let\PY@bf=\textbf\def\PY@tc##1{\textcolor[rgb]{0.67,0.13,1.00}{##1}}}

\def\PYZbs{\char`\\}
\def\PYZus{\char`\_}
\def\PYZob{\char`\{}
\def\PYZcb{\char`\}}
\def\PYZca{\char`\^}
\def\PYZam{\char`\&}
\def\PYZlt{\char`\<}
\def\PYZgt{\char`\>}
\def\PYZsh{\char`\#}
\def\PYZpc{\char`\%}
\def\PYZdl{\char`\$}
\def\PYZhy{\char`\-}
\def\PYZsq{\char`\'}
\def\PYZdq{\char`\"}
\def\PYZti{\char`\~}
% for compatibility with earlier versions
\def\PYZat{@}
\def\PYZlb{[}
\def\PYZrb{]}
\makeatother


    % Exact colors from NB
    \definecolor{incolor}{rgb}{0.0, 0.0, 0.5}
    \definecolor{outcolor}{rgb}{0.545, 0.0, 0.0}



    
\setcounter{secnumdepth}{0} % Turns off numbering for sections
\setlength{\parindent}{0cm}

    % Prevent overflowing lines due to hard-to-break entities
    \sloppy 
    % Setup hyperref package
    \hypersetup{
      breaklinks=true,  % so long urls are correctly broken across lines
      colorlinks=true,
      urlcolor=urlcolor,
      linkcolor=linkcolor,
      citecolor=citecolor,
      }
    % Slightly bigger margins than the latex defaults
    
    \geometry{verbose,tmargin=1in,bmargin=1in,lmargin=1in,rmargin=1in}
    
    


    \begin{document}
    
    
    \maketitle
    
    

    

    \hypertarget{jeffrey-elkner}{%
\subsection{Jeffrey Elkner}\label{jeffrey-elkner}}

\hypertarget{problem-point-pattern-descriptors}{%
\subsubsection{Problem: Point Pattern
Descriptors}\label{problem-point-pattern-descriptors}}

The tracking data for Hurricane Bill (2009) is given in the attached
package, suitable for ArcGIS. One folder contains the best track data
via SHIPS database (SHIPS\_2009Bill) and the other is a track
forecasting data from NHC (National Hurricane Center) for Hurricane Bill
at August 17 at 1500 (2009BillForecasting17\_1500) for forecasting
tracks in 12 hours interval for 3 days and 24 hour interval beyond that
for up to 5 days. Using the SHIPS data, please

\begin{quote}
\begin{enumerate}
\def\labelenumi{\alph{enumi}.}
\tightlist
\item
  Calculate the Standard Distance and output the SD value (3 points)
\end{enumerate}
\end{quote}

\vskip 0.2in

    \begin{Verbatim}[commandchars=\\\{\}]
{\color{incolor}In [{\color{incolor}1}]:} \PY{c+c1}{\PYZsh{} SOLUTION}
        \PY{c+c1}{\PYZsh{} http://webspace.ship.edu/pgmarr/Geo441/Examples/Standard\PYZpc{}20Distance.pdf}
        \PY{k+kn}{import} \PY{n+nn}{pandas} \PY{k}{as} \PY{n+nn}{pd}
        \PY{k+kn}{import} \PY{n+nn}{matplotlib}\PY{n+nn}{.}\PY{n+nn}{pyplot} \PY{k}{as} \PY{n+nn}{plt}\PY{p}{;} \PY{n}{plt}\PY{o}{.}\PY{n}{rcdefaults}\PY{p}{(}\PY{p}{)}
        \PY{k+kn}{from} \PY{n+nn}{matplotlib}\PY{n+nn}{.}\PY{n+nn}{patches} \PY{k}{import} \PY{n}{Circle}
        \PY{k+kn}{import} \PY{n+nn}{numpy} \PY{k}{as} \PY{n+nn}{np}
        \PY{k+kn}{import} \PY{n+nn}{thinkstats2} \PY{k}{as} \PY{n+nn}{ts}
        \PY{k+kn}{import} \PY{n+nn}{shapefile}
        \PY{k+kn}{from} \PY{n+nn}{ggs560}\PY{n+nn}{.}\PY{n+nn}{ggs560\PYZus{}tools} \PY{k}{import} \PY{n}{mean\PYZus{}center}\PY{p}{,} \PY{n}{standard\PYZus{}distance}
        
        \PY{k}{def} \PY{n+nf}{shape2dataframe}\PY{p}{(}\PY{n}{path2shape}\PY{p}{)}\PY{p}{:}
            \PY{n}{sf} \PY{o}{=} \PY{n}{shapefile}\PY{o}{.}\PY{n}{Reader}\PY{p}{(}\PY{n}{path2shape}\PY{p}{)}
        
            \PY{c+c1}{\PYZsh{}grab the shapefile\PYZsq{}s field names (omit the first psuedo field)}
            \PY{n}{fields} \PY{o}{=} \PY{p}{[}\PY{n}{x}\PY{p}{[}\PY{l+m+mi}{0}\PY{p}{]} \PY{k}{for} \PY{n}{x} \PY{o+ow}{in} \PY{n}{sf}\PY{o}{.}\PY{n}{fields}\PY{p}{]}\PY{p}{[}\PY{l+m+mi}{1}\PY{p}{:}\PY{p}{]}
            \PY{n}{records} \PY{o}{=} \PY{n}{sf}\PY{o}{.}\PY{n}{records}\PY{p}{(}\PY{p}{)}
            \PY{n}{shps} \PY{o}{=} \PY{p}{[}\PY{n}{s}\PY{o}{.}\PY{n}{points} \PY{k}{for} \PY{n}{s} \PY{o+ow}{in} \PY{n}{sf}\PY{o}{.}\PY{n}{shapes}\PY{p}{(}\PY{p}{)}\PY{p}{]}
        
            \PY{c+c1}{\PYZsh{}write the records into a dataframe}
            \PY{n}{shpdf} \PY{o}{=} \PY{n}{pd}\PY{o}{.}\PY{n}{DataFrame}\PY{p}{(}\PY{n}{columns}\PY{o}{=}\PY{n}{fields}\PY{p}{,} \PY{n}{data}\PY{o}{=}\PY{n}{records}\PY{p}{)}
        
            \PY{c+c1}{\PYZsh{}add the coordinate data to a column called \PYZdq{}coords\PYZdq{}}
            \PY{n}{shpdf} \PY{o}{=} \PY{n}{shpdf}\PY{o}{.}\PY{n}{assign}\PY{p}{(}\PY{n}{coords}\PY{o}{=}\PY{n}{shps}\PY{p}{)}
            
            \PY{k}{return} \PY{n}{shpdf}
        
        \PY{n}{df} \PY{o}{=} \PY{n}{shape2dataframe}\PY{p}{(}\PY{l+s+s1}{\PYZsq{}}\PY{l+s+s1}{Data/SHIPS\PYZus{}2009Bill/SHIPS\PYZus{}2009Bill}\PY{l+s+s1}{\PYZsq{}}\PY{p}{)}
        
        \PY{k}{def} \PY{n+nf}{calc\PYZus{}and\PYZus{}display\PYZus{}standard\PYZus{}distance}\PY{p}{(}\PY{n}{df}\PY{p}{,} \PY{n}{title}\PY{o}{=}\PY{l+s+s1}{\PYZsq{}}\PY{l+s+s1}{\PYZsq{}}\PY{p}{)}\PY{p}{:}
            \PY{n}{xs} \PY{o}{=} \PY{p}{[}\PY{n}{point}\PY{p}{[}\PY{l+m+mi}{0}\PY{p}{]}\PY{p}{[}\PY{l+m+mi}{0}\PY{p}{]} \PY{k}{for} \PY{n}{point} \PY{o+ow}{in} \PY{n}{df}\PY{o}{.}\PY{n}{coords}\PY{p}{]}
            \PY{n}{ys} \PY{o}{=} \PY{p}{[}\PY{n}{point}\PY{p}{[}\PY{l+m+mi}{0}\PY{p}{]}\PY{p}{[}\PY{l+m+mi}{1}\PY{p}{]} \PY{k}{for} \PY{n}{point} \PY{o+ow}{in} \PY{n}{df}\PY{o}{.}\PY{n}{coords}\PY{p}{]}
            \PY{n}{cx}\PY{p}{,} \PY{n}{cy} \PY{o}{=} \PY{n}{mean\PYZus{}center}\PY{p}{(}\PY{n+nb}{list}\PY{p}{(}\PY{n+nb}{zip}\PY{p}{(}\PY{n}{xs}\PY{p}{,} \PY{n}{ys}\PY{p}{)}\PY{p}{)}\PY{p}{)}
            \PY{n}{sd} \PY{o}{=} \PY{n}{standard\PYZus{}distance}\PY{p}{(}\PY{n}{xs}\PY{p}{,} \PY{n}{ys}\PY{p}{)}
        
            \PY{c+c1}{\PYZsh{} Print out the mean center and standard distance}
            \PY{n}{s} \PY{o}{=} \PY{p}{(}\PY{l+s+s2}{\PYZdq{}}\PY{l+s+s2}{The standard distance is }\PY{l+s+si}{\PYZob{}:0.2f\PYZcb{}}\PY{l+s+s2}{ from mean center }\PY{l+s+s2}{\PYZdq{}}
                 \PY{l+s+s2}{\PYZdq{}}\PY{l+s+s2}{of (}\PY{l+s+si}{\PYZob{}:0.2f\PYZcb{}}\PY{l+s+s2}{, }\PY{l+s+si}{\PYZob{}:0.2f\PYZcb{}}\PY{l+s+s2}{).}\PY{l+s+s2}{\PYZdq{}}
                \PY{p}{)}
            \PY{n+nb}{print}\PY{p}{(}\PY{n}{s}\PY{o}{.}\PY{n}{format}\PY{p}{(}\PY{n}{sd}\PY{p}{,} \PY{n}{cx}\PY{p}{,} \PY{n}{cy}\PY{p}{)}\PY{p}{)}
        
            \PY{n}{plt}\PY{o}{.}\PY{n}{scatter}\PY{p}{(}\PY{n}{xs}\PY{p}{,} \PY{n}{ys}\PY{p}{)}
            \PY{n}{plt}\PY{o}{.}\PY{n}{plot}\PY{p}{(}\PY{p}{[}\PY{n}{cx}\PY{p}{]}\PY{p}{,} \PY{p}{[}\PY{n}{cy}\PY{p}{]}\PY{p}{,} \PY{l+s+s1}{\PYZsq{}}\PY{l+s+s1}{ro}\PY{l+s+s1}{\PYZsq{}}\PY{p}{)}
            \PY{n}{plt}\PY{o}{.}\PY{n}{gca}\PY{p}{(}\PY{p}{)}\PY{o}{.}\PY{n}{annotate}\PY{p}{(}\PY{l+s+s1}{\PYZsq{}}\PY{l+s+s1}{MC}\PY{l+s+s1}{\PYZsq{}}\PY{p}{,} \PY{n}{xy}\PY{o}{=}\PY{p}{(}\PY{n}{cx}\PY{p}{,} \PY{n}{cy}\PY{p}{)}\PY{p}{,} \PY{n}{xytext}\PY{o}{=}\PY{p}{(}\PY{n}{cx} \PY{o}{+} \PY{l+m+mi}{1}\PY{p}{,} \PY{n}{cy} \PY{o}{+} \PY{l+m+mi}{1}\PY{p}{)}\PY{p}{)}
            \PY{n}{plt}\PY{o}{.}\PY{n}{gca}\PY{p}{(}\PY{p}{)}\PY{o}{.}\PY{n}{set\PYZus{}aspect}\PY{p}{(}\PY{l+s+s1}{\PYZsq{}}\PY{l+s+s1}{equal}\PY{l+s+s1}{\PYZsq{}}\PY{p}{)}
            \PY{n}{circle} \PY{o}{=} \PY{n}{plt}\PY{o}{.}\PY{n}{Circle}\PY{p}{(}\PY{p}{(}\PY{n}{cx}\PY{p}{,} \PY{n}{cy}\PY{p}{)}\PY{p}{,} \PY{n}{sd}\PY{p}{,} \PY{n}{color}\PY{o}{=}\PY{l+s+s1}{\PYZsq{}}\PY{l+s+s1}{red}\PY{l+s+s1}{\PYZsq{}}\PY{p}{,} \PY{n}{alpha}\PY{o}{=}\PY{l+m+mf}{0.2}\PY{p}{)}
            \PY{n}{plt}\PY{o}{.}\PY{n}{gca}\PY{p}{(}\PY{p}{)}\PY{o}{.}\PY{n}{add\PYZus{}artist}\PY{p}{(}\PY{n}{circle}\PY{p}{)}
            \PY{n}{plt}\PY{o}{.}\PY{n}{title}\PY{p}{(}\PY{n}{title}\PY{p}{)}
            \PY{n}{plt}\PY{o}{.}\PY{n}{xlabel}\PY{p}{(}\PY{l+s+s1}{\PYZsq{}}\PY{l+s+s1}{Longitute}\PY{l+s+s1}{\PYZsq{}}\PY{p}{)}
            \PY{n}{plt}\PY{o}{.}\PY{n}{ylabel}\PY{p}{(}\PY{l+s+s1}{\PYZsq{}}\PY{l+s+s1}{Latitude}\PY{l+s+s1}{\PYZsq{}}\PY{p}{)}
            \PY{n}{plt}\PY{o}{.}\PY{n}{show}\PY{p}{(}\PY{p}{)}
            
        \PY{n}{calc\PYZus{}and\PYZus{}display\PYZus{}standard\PYZus{}distance}\PY{p}{(}\PY{n}{df}\PY{p}{,} \PY{l+s+s1}{\PYZsq{}}\PY{l+s+s1}{Path of Hurricane Bill}\PY{l+s+s1}{\PYZsq{}}\PY{p}{)}
\end{Verbatim}


    \begin{Verbatim}[commandchars=\\\{\}]
The standard distance is 16.16 from mean center of (-53.87, 23.10).

    \end{Verbatim}

    \begin{center}
    \adjustimage{max size={0.9\linewidth}{0.9\paperheight}}{output_1_1.png}
    \end{center}
    { \hspace*{\fill} \\}
    

    \begin{quote}
\begin{enumerate}
\def\labelenumi{\alph{enumi}.}
\setcounter{enumi}{1}
\tightlist
\item
  Calculate the deviational ellipse (Directional Distribution) and
  output the rotation angle, and the deviations along the major axis and
  minor axis. (3 points)
\end{enumerate}
\end{quote}

\vskip 0.2in

    \begin{Verbatim}[commandchars=\\\{\}]
{\color{incolor}In [{\color{incolor}2}]:} \PY{c+c1}{\PYZsh{} SOLUTION}
        \PY{c+c1}{\PYZsh{} http://tinyurl.com/standarddeviationalellipse}
        \PY{c+c1}{\PYZsh{} http://tinyurl.com/standarddeviationalellipse2}
        \PY{c+c1}{\PYZsh{} http://tinyurl.com/standarddeviationalellipse3}
        \PY{c+c1}{\PYZsh{} http://blog.mmast.net/conics\PYZhy{}matplotlib}
        
        \PY{k}{def} \PY{n+nf}{directional\PYZus{}distribution\PYZus{}analysis}\PY{p}{(}\PY{n}{df}\PY{p}{)}\PY{p}{:}
            \PY{n}{xs} \PY{o}{=} \PY{p}{[}\PY{n}{point}\PY{p}{[}\PY{l+m+mi}{0}\PY{p}{]}\PY{p}{[}\PY{l+m+mi}{0}\PY{p}{]} \PY{k}{for} \PY{n}{point} \PY{o+ow}{in} \PY{n}{df}\PY{o}{.}\PY{n}{coords}\PY{p}{]}
            \PY{n}{ys} \PY{o}{=} \PY{p}{[}\PY{n}{point}\PY{p}{[}\PY{l+m+mi}{0}\PY{p}{]}\PY{p}{[}\PY{l+m+mi}{1}\PY{p}{]} \PY{k}{for} \PY{n}{point} \PY{o+ow}{in} \PY{n}{df}\PY{o}{.}\PY{n}{coords}\PY{p}{]}
            \PY{n}{cx}\PY{p}{,} \PY{n}{cy} \PY{o}{=} \PY{n}{mean\PYZus{}center}\PY{p}{(}\PY{n+nb}{list}\PY{p}{(}\PY{n+nb}{zip}\PY{p}{(}\PY{n}{xs}\PY{p}{,} \PY{n}{ys}\PY{p}{)}\PY{p}{)}\PY{p}{)}
        
            \PY{c+c1}{\PYZsh{} Move points to center at mean center}
            \PY{n}{xps} \PY{o}{=} \PY{p}{[}\PY{n}{x\PYZus{}i} \PY{o}{\PYZhy{}} \PY{n}{cx} \PY{k}{for} \PY{n}{x\PYZus{}i} \PY{o+ow}{in} \PY{n}{xs}\PY{p}{]}
            \PY{n}{yps} \PY{o}{=} \PY{p}{[}\PY{n}{y\PYZus{}i} \PY{o}{\PYZhy{}} \PY{n}{cy} \PY{k}{for} \PY{n}{y\PYZus{}i} \PY{o+ow}{in} \PY{n}{ys}\PY{p}{]}
        
            \PY{c+c1}{\PYZsh{} Calulate A, B, and C}
            \PY{n}{A} \PY{o}{=} \PY{n+nb}{sum}\PY{p}{(}\PY{p}{[}\PY{n}{x\PYZus{}i}\PY{o}{*}\PY{o}{*}\PY{l+m+mi}{2} \PY{k}{for} \PY{n}{x\PYZus{}i} \PY{o+ow}{in} \PY{n}{xps}\PY{p}{]}\PY{p}{)} \PY{o}{\PYZhy{}} \PY{n+nb}{sum}\PY{p}{(}\PY{p}{[}\PY{n}{y\PYZus{}i}\PY{o}{*}\PY{o}{*}\PY{l+m+mi}{2} \PY{k}{for} \PY{n}{y\PYZus{}i} \PY{o+ow}{in} \PY{n}{yps}\PY{p}{]}\PY{p}{)}
        
            \PY{c+c1}{\PYZsh{} compute the sum of products of moved points}
            \PY{n}{sop} \PY{o}{=} \PY{n+nb}{sum}\PY{p}{(}\PY{p}{[}\PY{n}{xps}\PY{p}{[}\PY{n}{i}\PY{p}{]} \PY{o}{*} \PY{n}{yps}\PY{p}{[}\PY{n}{i}\PY{p}{]} \PY{k}{for} \PY{n}{i} \PY{o+ow}{in} \PY{n+nb}{range}\PY{p}{(}\PY{n+nb}{len}\PY{p}{(}\PY{n}{xps}\PY{p}{)}\PY{p}{)}\PY{p}{]}\PY{p}{)}
        
            \PY{n}{B} \PY{o}{=} \PY{p}{(}\PY{n}{A}\PY{o}{*}\PY{o}{*}\PY{l+m+mi}{2} \PY{o}{+} \PY{l+m+mi}{4} \PY{o}{*} \PY{n}{sop}\PY{o}{*}\PY{o}{*}\PY{l+m+mi}{2}\PY{p}{)}\PY{o}{*}\PY{o}{*}\PY{l+m+mf}{0.5}
            \PY{n}{C} \PY{o}{=} \PY{l+m+mi}{2} \PY{o}{*} \PY{n}{sop}
        
            \PY{c+c1}{\PYZsh{} Calculate the angle of rotation}
            \PY{n}{tan\PYZus{}theta} \PY{o}{=} \PY{p}{(}\PY{n}{A} \PY{o}{+} \PY{n}{B}\PY{p}{)} \PY{o}{/} \PY{n}{C}
            \PY{n}{theta} \PY{o}{=} \PY{n}{np}\PY{o}{.}\PY{n}{arctan}\PY{p}{(}\PY{n}{tan\PYZus{}theta}\PY{p}{)}
            \PY{n}{s} \PY{o}{=} \PY{l+s+s2}{\PYZdq{}}\PY{l+s+s2}{The angle of rotation is }\PY{l+s+si}{\PYZob{}:0.2f\PYZcb{}}\PY{l+s+s2}{ radians or }\PY{l+s+si}{\PYZob{}:0.2f\PYZcb{}}\PY{l+s+s2}{ degrees.}\PY{l+s+s2}{\PYZdq{}}
            \PY{n+nb}{print}\PY{p}{(}\PY{n}{s}\PY{o}{.}\PY{n}{format}\PY{p}{(}\PY{n}{theta}\PY{p}{,} \PY{n}{np}\PY{o}{.}\PY{n}{degrees}\PY{p}{(}\PY{p}{[}\PY{n}{theta}\PY{p}{]}\PY{p}{)}\PY{p}{[}\PY{l+m+mi}{0}\PY{p}{]}\PY{p}{)}\PY{p}{)}
        
            \PY{c+c1}{\PYZsh{} Calculate deviations along major and minor axes}
            \PY{c+c1}{\PYZsh{} Precalculate some values}
            \PY{n}{n} \PY{o}{=} \PY{n+nb}{len}\PY{p}{(}\PY{n}{xps}\PY{p}{)}
            \PY{n}{cos\PYZus{}theta} \PY{o}{=} \PY{n}{np}\PY{o}{.}\PY{n}{cos}\PY{p}{(}\PY{n}{theta}\PY{p}{)}
            \PY{n}{sin\PYZus{}theta} \PY{o}{=} \PY{n}{np}\PY{o}{.}\PY{n}{sin}\PY{p}{(}\PY{n}{theta}\PY{p}{)}
        
            \PY{c+c1}{\PYZsh{} Calculate x\PYZhy{}axis deviation}
            \PY{n}{numeratorSigmaX} \PY{o}{=}  \PY{n+nb}{sum}\PY{p}{(}\PY{p}{[}\PY{p}{(}\PY{n}{xps}\PY{p}{[}\PY{n}{i}\PY{p}{]} \PY{o}{*} \PY{n}{cos\PYZus{}theta} \PY{o}{\PYZhy{}}
                                     \PY{n}{yps}\PY{p}{[}\PY{n}{i}\PY{p}{]} \PY{o}{*} \PY{n}{sin\PYZus{}theta}\PY{p}{)}\PY{o}{*}\PY{o}{*}\PY{l+m+mi}{2} \PY{k}{for} \PY{n}{i} \PY{o+ow}{in} \PY{n+nb}{range}\PY{p}{(}\PY{n}{n}\PY{p}{)}\PY{p}{]}\PY{p}{)}
            \PY{n}{SigmaX} \PY{o}{=} \PY{l+m+mi}{2}\PY{o}{*}\PY{o}{*}\PY{l+m+mf}{0.5} \PY{o}{*} \PY{p}{(}\PY{n}{numeratorSigmaX} \PY{o}{/} \PY{n}{n}\PY{p}{)}\PY{o}{*}\PY{o}{*}\PY{l+m+mf}{0.5}
        
            \PY{c+c1}{\PYZsh{} Calculate y\PYZhy{}axis deviation}
            \PY{n}{numeratorSigmaY} \PY{o}{=}  \PY{n+nb}{sum}\PY{p}{(}\PY{p}{[}\PY{p}{(}\PY{n}{xps}\PY{p}{[}\PY{n}{i}\PY{p}{]} \PY{o}{*} \PY{n}{sin\PYZus{}theta} \PY{o}{+}
                                     \PY{n}{yps}\PY{p}{[}\PY{n}{i}\PY{p}{]} \PY{o}{*} \PY{n}{cos\PYZus{}theta}\PY{p}{)}\PY{o}{*}\PY{o}{*}\PY{l+m+mi}{2} \PY{k}{for} \PY{n}{i} \PY{o+ow}{in} \PY{n+nb}{range}\PY{p}{(}\PY{n}{n}\PY{p}{)}\PY{p}{]}\PY{p}{)}
            \PY{n}{SigmaY} \PY{o}{=} \PY{l+m+mi}{2}\PY{o}{*}\PY{o}{*}\PY{l+m+mf}{0.5} \PY{o}{*} \PY{p}{(}\PY{n}{numeratorSigmaY} \PY{o}{/} \PY{n}{n}\PY{p}{)}\PY{o}{*}\PY{o}{*}\PY{l+m+mf}{0.5}
        
            \PY{n}{s} \PY{o}{=} \PY{l+s+s2}{\PYZdq{}}\PY{l+s+s2}{The axes deviations are x\PYZhy{}axis: }\PY{l+s+si}{\PYZob{}:0.2f\PYZcb{}}\PY{l+s+s2}{ and y\PYZhy{}axis: }\PY{l+s+si}{\PYZob{}:0.2f\PYZcb{}}\PY{l+s+s2}{.}\PY{l+s+s2}{\PYZdq{}}
            \PY{n+nb}{print}\PY{p}{(}\PY{n}{s}\PY{o}{.}\PY{n}{format}\PY{p}{(}\PY{n}{SigmaX}\PY{p}{,} \PY{n}{SigmaY}\PY{p}{)}\PY{p}{)}
            
            \PY{k}{return} \PY{n}{xs}\PY{p}{,} \PY{n}{ys}\PY{p}{,} \PY{n}{cx}\PY{p}{,} \PY{n}{cy}\PY{p}{,} \PY{n}{xps}\PY{p}{,} \PY{n}{yps}\PY{p}{,} \PY{n}{SigmaX}\PY{p}{,} \PY{n}{SigmaY} 
        
        
        \PY{n}{xs}\PY{p}{,} \PY{n}{ys}\PY{p}{,} \PY{n}{cx}\PY{p}{,} \PY{n}{cy}\PY{p}{,} \PY{n}{xps}\PY{p}{,} \PY{n}{yps}\PY{p}{,} \PY{n}{SigmaX}\PY{p}{,} \PY{n}{SigmaY} \PY{o}{=} \PY{n}{directional\PYZus{}distribution\PYZus{}analysis}\PY{p}{(}\PY{n}{df}\PY{p}{)}
\end{Verbatim}


    \begin{Verbatim}[commandchars=\\\{\}]
The angle of rotation is -0.82 radians or -46.90 degrees.
The axes deviations are x-axis: 8.12 and y-axis: 21.37.

    \end{Verbatim}


    \begin{quote}
\begin{enumerate}
\def\labelenumi{\alph{enumi}.}
\setcounter{enumi}{2}
\tightlist
\item
  Show that the total circular variance is conserved before and after
  the rotation or the conservation relationship between the standard
  distance and the semi-major and semi-minor axes of the deviational
  ellipse. (3 points)
\end{enumerate}
\end{quote}

\vskip 0.2in


    \emph{Solution}: From slide 17 in our notes, we will verifying that
\(2(\sigma_x^2 + \sigma_y^2) = 2SD^2\).

\vskip 0.2in

    \begin{Verbatim}[commandchars=\\\{\}]
{\color{incolor}In [{\color{incolor}3}]:} \PY{n}{var1} \PY{o}{=} \PY{n}{SigmaX}\PY{o}{*}\PY{o}{*}\PY{l+m+mi}{2} \PY{o}{+} \PY{n}{SigmaY}\PY{o}{*}\PY{o}{*}\PY{l+m+mi}{2}
        \PY{n}{xssq} \PY{o}{=} \PY{n+nb}{sum}\PY{p}{(}\PY{p}{[}\PY{p}{(}\PY{n}{x} \PY{o}{\PYZhy{}} \PY{n}{cx}\PY{p}{)}\PY{o}{*}\PY{o}{*}\PY{l+m+mi}{2} \PY{k}{for} \PY{n}{x} \PY{o+ow}{in} \PY{n}{xs}\PY{p}{]}\PY{p}{)}
        \PY{n}{yssq} \PY{o}{=} \PY{n+nb}{sum}\PY{p}{(}\PY{p}{[}\PY{p}{(}\PY{n}{y} \PY{o}{\PYZhy{}} \PY{n}{cy}\PY{p}{)}\PY{o}{*}\PY{o}{*}\PY{l+m+mi}{2} \PY{k}{for} \PY{n}{y} \PY{o+ow}{in} \PY{n}{ys}\PY{p}{]}\PY{p}{)}
        \PY{n}{var2} \PY{o}{=} \PY{l+m+mi}{2} \PY{o}{*} \PY{p}{(}\PY{n}{xssq} \PY{o}{+} \PY{n}{yssq}\PY{p}{)} \PY{o}{/} \PY{n+nb}{len}\PY{p}{(}\PY{n}{xs}\PY{p}{)}
        
        \PY{n+nb}{print}\PY{p}{(}\PY{l+s+s1}{\PYZsq{}}\PY{l+s+si}{\PYZob{}:0.2f\PYZcb{}}\PY{l+s+s1}{ equals }\PY{l+s+si}{\PYZob{}:0.2f\PYZcb{}}\PY{l+s+s1}{?}\PY{l+s+s1}{\PYZsq{}}\PY{o}{.}\PY{n}{format}\PY{p}{(}\PY{n}{var1}\PY{p}{,} \PY{n}{var2}\PY{p}{)}\PY{p}{)}
\end{Verbatim}


    \begin{Verbatim}[commandchars=\\\{\}]
522.56 equals 522.56?

    \end{Verbatim}


    The results are equal and we have shown that circular variance is
conserved.

\vskip 0.2in


    \begin{quote}
\begin{enumerate}
\def\labelenumi{\alph{enumi}.}
\setcounter{enumi}{3}
\tightlist
\item
  Use the forecasting data to repeat the Directional Distribution
  analysis. (1 point)
\end{enumerate}
\end{quote}

\vskip 0.2in

    \begin{Verbatim}[commandchars=\\\{\}]
{\color{incolor}In [{\color{incolor}4}]:} \PY{c+c1}{\PYZsh{} SOLUTION}
        \PY{n}{df2} \PY{o}{=} \PY{n}{shape2dataframe}\PY{p}{(}\PY{l+s+s1}{\PYZsq{}}\PY{l+s+s1}{Data/BillForcast/2009Bill\PYZus{}Forecasting17\PYZus{}1500}\PY{l+s+s1}{\PYZsq{}}\PY{p}{)}
        \PY{n}{calc\PYZus{}and\PYZus{}display\PYZus{}standard\PYZus{}distance}\PY{p}{(}\PY{n}{df2}\PY{p}{,} \PY{l+s+s1}{\PYZsq{}}\PY{l+s+s1}{Forcasted Path of Hurricane Bill}\PY{l+s+s1}{\PYZsq{}}\PY{p}{)}
        \PY{n}{results} \PY{o}{=} \PY{n}{directional\PYZus{}distribution\PYZus{}analysis}\PY{p}{(}\PY{n}{df2}\PY{p}{)}
\end{Verbatim}


    \begin{Verbatim}[commandchars=\\\{\}]
The standard distance is 9.28 from mean center of (-55.12, 19.74).

    \end{Verbatim}

    \begin{center}
    \adjustimage{max size={0.9\linewidth}{0.9\paperheight}}{output_9_1.png}
    \end{center}
    { \hspace*{\fill} \\}
    
    \begin{Verbatim}[commandchars=\\\{\}]
The angle of rotation is -0.90 radians or -51.63 degrees.
The axes deviations are x-axis: 1.87 and y-axis: 12.99.

    \end{Verbatim}


    \begin{quote}
\begin{enumerate}
\def\labelenumi{\alph{enumi}.}
\setcounter{enumi}{4}
\tightlist
\item
  (Optional/open-end) Please compare the results of the directional
  distribution from the SHIPS data and from the forecasting data and
  discuss the possibility for using the forecasting data to cluster
  tracks of tropical cyclones. (xx points)
\end{enumerate}
\end{quote}

\vskip 0.2in


    \emph{SOLUTION}: The mean center and angle of rotation of both the
actual path in the SHIPS data and that of the forcast appear fairly
close, while the minor axis of the elipse of the actual path of
Hurricane Bill is larger than that of the forcaste. This is a result of
the larger curvature of the actual path the hurricane followed.

\vskip 0.2in


    \hypertarget{notes-for-additional-information}{%
\paragraph{Notes for additional
information:}\label{notes-for-additional-information}}

\begin{enumerate}
\def\labelenumi{\arabic{enumi}.}
\item
  The SHIPS data files can be found at
  ftp://rammftp.cira.colostate.edu/demaria/SHIPS/. One of the data files
  for the period 1989-2009 hurricane season can be downloaded at
  ftp://rammftp.cira.colostate.edu/demaria/SHIPS/2010/lsdiaga\_1982\_2009\_rean\_biascorr\_sat.dat,
  and the corresponding document file for this data file is also
  available at
  ftp://rammftp.cira.colostate.edu/demaria/SHIPS/2009/SHIPS\_predictor\_file\_2009.doc.
  The original data are in ASCII but the data are hard to handle.
\item
  The forecasting data can be found at http://www.nhc.noaa.gov/data/.
  More information including the metadata information can also be found
  in the same site.
\end{enumerate}

\vskip 0.2in


    % Add a bibliography block to the postdoc
    
    
    
    \end{document}
